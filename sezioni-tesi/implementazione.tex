\chapter{Implementazione del sistema di protezione}
Nello svolgimento di questa tesi, abbiamo implementato sul nostro sistema una versione modificata di KAISER per proteggere il nucleo da Meltdown. 
Per semplicità, è stato protetta soltanto la sezione \emph{sistema/condivisa} della memoria virtuale, contenente la finestra di memoria virtuale (vedi sezione \vref{sec:nucleo-memoria}).
Le sezioni io/condivisa e sistema/privata sono dunque ancora vulnerabili a Meltdown.

\section{La finestra di memoria fisica}
Nel nostro sistema, la finestra di memoria virtuale occupa interamente la \emph{prima} entrata della tabella di livello 4 di ogni processo ed è inizializzata dal boot loader all'avvio della macchina virtuale.
Grazie a questa proprietà, abbiamo potuto effettuare una prima ottimizzazione rispetto alla versione di KAISER proposta da \textcite{gruss:kaslr}: invece di creare lo spazio d'indirizzamento shadow a partire dalla tabella di livello 4, nel nostro sistema viene costruita a partire dalla \emph{tabella di livello 3}.
Al momento del passaggio nelle funzioni trampolino, il sistema modificherà la prima entrata della tabella di livello 4 del processo in esecuzione invece del registro CR4, inserendovi l'indirizzo della tabella di livello 3 "kernel" (se il processore sta passando a livello sistema) o "shadow" (se sta tornando a livello utente).

Oltre al risparmio di spazio per avere una tabella duplicata in meno, questa ottimizzazione evita lo svuotamento implicito del \emph{Translation Lookaside Buffer} (TLB) dovuto alla modifica del registro CR4~\cite{gruss:kaslr}, che avrebbe un impatto negativo sulle prestazioni. 
Nella proposta di \textcite{gruss:kaslr}, questo problema veniva aggirato sfruttando alcune funzionalità delle CPU moderne di cui non disponiamo nel nostro sistema emulato.

\section{Le funzioni e la memoria trampolino}
Nel paragrafo \vref{sec:kaiser} abbiamo affermato che alcune porzioni del kernel devono essere mappate nello spazio di indirizzamento shadow per permettere il funzionamento delle interruzioni.
Nella nostra implementazione le porzioni necessarie (che abbiamo denominato nel loro complesso come \emph{memoria trampolino})sono state raccolte in tre sezioni Assembly: una contenente il codice (\texttt{.trampoline\_text}) e due contenenti i dati (\texttt{.trampoline\_data} per le variabili non costanti e \texttt{.trampoline\_bss} per le costanti).
Ogni sezione è stato allineata alla dimensione delle pagine virtuali (4KiB), in modo che le pagine in cui si trovano le tre sezioni siano occupate da esse in maniera esclusiva e non vi si trovino altre porzioni del kernel.
Questo ci permette di inserirne la traduzione da indirizzo virtuale a fisico nello spazio di indirizzamento shadow.

Le funzioni trampolino d'ingresso (nel kernel) e di uscita (dal kernel) si occupano di aggiornare la prima entrata della tabella di livello 4 con il descrittore di tabella 3 opportuno.
Mentre la tabella di livello 3 \emph{kernel} viene creata dal boot loader, la tabella di livello 3 \emph{shadow} viene creata dal nostro programma durante la fase di inizializzazione della memoria virtuale (vedi sezione \vref{sec:memoria-shadow}) e il suo descrittore viene conservato nella variabile globale \texttt{des\_finestra\_shadow} (riga 754 del listato \ref{lst:sistemacpp-743}).
La sostituzione dello spazio di indirizzamento si limita in pratica a scrivere nella prima entrata della tabella di livello 4 del processo in esecuzione il contenuto di \texttt{des\_finestra\_shadow}, se stiamo passando al livello utente, o di \texttt{des\_finestra\_kernel} (inizializzato da noi; riga 1725 del listato \ref{lst:sistemas-685}), se stiamo passando al livello sistema.

\section{Costruzione dello spazio di memoria shadow}
\label{sec:memoria-shadow}

\section{La gestione dei TSS}

\section{Il TLB}
Nonostante si sia evitato lo svuotamento implicito del TLB, è in ogni caso necessario invalidarlo forzatamente quando il sistema passa da sistema a utente. 
L'efficacia di KAISER si basa sulla garanzia che nell'albero di traduzione di ogni processo non vi sia il kernel e la finestra di memoria fisica (o la struttura equivalente per lo specifico sistema operativo) quando il processore lavora a livello utente e che \emph{la CPU non abbia nessun altro modo per ottenere le traduzioni degli indirizzi}.
Quando il processore lavora in modalità privilegiata, può accedere all'albero di traduzione completo di finestra di memoria fisica e la traduzione degli indirizzi a cui accede viene conservata nel TLB.

Se gli indirizzi della finestra di memoria non venissero invalidati quando il processore torna a livello utente, un processo attaccante potrebbe accedere tranquillamente agli indirizzi di sistema acceduti dal processore, in quanto, essendo le loro traduzioni conservate nel TLB, il processore \emph{non} utilizzerà l'albero di traduzione, bypassando così la protezione contro Meltdown.

Dunque, è necessario invalidare il TLB nella funzione trampolino di uscita dal kernel (vedi riga 565 del listato \vref{lst:sistemas-519}).